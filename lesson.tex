\documentclass{article}

\usepackage{amsmath}
\usepackage{amssymb}
\usepackage{tikz}
\usepackage[margin=2cm]{geometry}
\usepackage[super]{nth}
\usepackage{calc}
\usepackage{enumitem}
\usepackage{setspace}
\usepackage[utf8]{inputenc}

\usepackage[hidelinks, pagebackref, bookmarksopen, bookmarksnumbered]{hyperref}

\pgfdeclarelayer{bg}
\pgfsetlayers{bg,main}

\usepackage{esvect}

\usetikzlibrary{arrows,positioning,shapes,fit,calc}

\title{Mathematical Foundations for Software Engineering \\[1ex] \large Course notes}
\author{Attila Matolcsy}
\date{\nth{25} August 2023 - Present}

\let\stdsection\section
\renewcommand\section{\newpage\stdsection}

\usepackage[parfill]{parskip}

\begin{document}
\maketitle

\tableofcontents

\section{About}
This document is Attila Matolcsy's own personal notes from the University of Gothenburg's Software Engineering and Management Bsc. Programme's Mathematical Foundations for SEM course.

These are all the notes from lessons, unprocessed.

\begin{minipage}{\linewidth}
  \Large WARNING \vspace{.25cm}
\end{minipage}
This document has typos in it, please use the main document that is cleaned up!

\section{\date{\nth{28} August 2023}}

\subsection{Introduction to the course}

Lessons are not mandatory, nor the TA sessions.
We join a TA group on Canvas, we are asked not to jump between them. The gropus should not have more than 8 persons / group.

Workload $\approx 200 h$

Groups up to 3 are allowed but submissions must be on an individual basis.

FAQ is available on Canvas

You can send christian.berger@gu.se an email about problems that come up during class.

Course literature is available on Canvas, non of which is mandatory.

\subsection{Logic}

Logic = Meaning of mathematical statements $\wedge$ basis of reasoning

Applications = design of computing machines

$0$s $\And$ $1$s.

Proofs = matheatical argument, essential for programs \\ = Security of systems

Theorem = Proven mathematical statements

Propositions = declerative statemnt that is either True or False

For e.g.: \\
- Stockholm is the capital of Sweden \\
- $4x5 = 20$ \\
- $\pi \approx \frac{22}{7}$

Counter e.g.: \\
- attend my lectures \\
- $x + 1 = \pi$

Propositions are named by letters: $p, q, r, s$

field of propositional logic = the field that deals with propositionals

mathematical statements can be compined $\Rightarrow$ compoind propositions

Def1: $\neg p = $ not p

Def2: Conjuction: $p \and q$

\begin{tabular}{cc|c|c}
  $p$ & $q$ & $p \wedge q$  & $\neg \left( A \wedge B \right)$  \\ \hline
  $T$ & $T$ & $T$ & $F$ \\
  $F$ & $T$ & $F$ & $T$ \\
  $T$ & $F$ & $F$ & $T$ \\
  $F$ & $F$ & $F$ & $T$
\end{tabular}

(From now on I will refer to Trues ($T$s) as $1$ and Falses ($F$s) as $0$)

Def 3: $\vee$ is logical or\\
\begin{tabular}{cc|c}
  $p$ & $q$ & $p \vee q$ \\ \hline
  $1$ & $1$ & $1$ \\
  $1$ & $0$ & $1$ \\
  $0$ & $1$ & $1$ \\
  $0$ & $0$ & $0$
\end{tabular}

Def 4: exclusive or\\
\begin{tabular}{cc|c|c}
  $p$ & $q$ & $p \oplus q$ & $\neg \left( p \oplus q \right)$ \\ \hline
  $1$ & $1$ & $0$ & $1$ \\
  $1$ & $0$ & $1$ & $0$ \\
  $0$ & $1$ & $1$ & $0$ \\
  $0$ & $0$ & $0$ & $1$
\end{tabular}

Def 5: implication / conditional statements \\
\begin{tabular}{cc|c||c||c}
  $p$ & $q$ & $p \Rightarrow q$ & $\neg p \vee q$ & $\neg p$ \\ \hline
  $1$ & $1$ & $1$ & $1$ & $0$ \\
  $1$ & $0$ & $0$ & $0$ & $0$ \\
  $0$ & $1$ & $1$ & $1$ & $1$ \\
  $0$ & $0$ & $1$ & $1$ & $1$
\end{tabular}

Def 6: Biconditional statements, if and only if, iff\\
\begin{tabular}{cc|c}
  $p$ & $q$ & $p \Leftrightarrow q$ \\ \hline
  $1$ & $1$ & $1$ \\
  $1$ & $0$ & $0$ \\
  $0$ & $1$ & $0$ \\
  $0$ & $0$ & $1$
\end{tabular}

Def 7:
tantology: always true regardless of values of the propositions

Contardiction: always false regardless of values of the propositions

contingency: $\neg \left(\text{tantology}\right) \wedge \neg \left(\text{contradiction}\right)$

equivalents:\\
\begin{tabular}{ccc}
  $p \wedge 1$ & $\equiv$ & $p$ \\
  $p \vee 1$ & $\equiv$ & $1$ \\
  $p \wedge 0$ & $\equiv$ & $0$ \\
  $p \vee 0$ & $\equiv$ & $p$
\end{tabular}

$p \wedge q \equiv p$\\
\begin{tabular}{cc|c}
  $p$ & $q$ & $p \wedge q$ \\ \hline
  $1$ & $1$ & $1$ \\
  $0$ & $0$ & $0$ \\
\end{tabular}

negation laws:\\
$p \vee \neg p \equiv 1$ \\
\begin{tabular}{cc|c}
  $p$ & $\neg p$ & $p \vee \neg p$ \\ \hline
  $1$ & $0$ & $1$ \\
  $0$ & $1$ & $1$ \\
\end{tabular}

$p \wedge \neg p \equiv 0$ \\
\begin{tabular}{cc|c}
  $p$ & $\neg p$ & $p \wedge \neg p$ \\ \hline
  $1$ & $0$ & $0$ \\
  $0$ & $1$ & $0$ \\
\end{tabular}

\end{document}