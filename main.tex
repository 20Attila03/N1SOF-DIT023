\documentclass{article}

\usepackage{amsmath}
\usepackage{amssymb}
\usepackage{tikz}
\usepackage[margin=2cm]{geometry}
\usepackage[super]{nth}
\usepackage{calc}
\usepackage{enumitem}
\usepackage{setspace}
\usepackage[utf8]{inputenc}

\usepackage[hidelinks, pagebackref, bookmarksopen, bookmarksnumbered]{hyperref}

\pgfdeclarelayer{bg}
\pgfsetlayers{bg,main}

\usepackage{esvect}

\usetikzlibrary{arrows,positioning,shapes,fit,calc}

\title{Mathematical Foundations for Software Engineering \\[1ex] \large Course notes}
\author{Attila Matolcsy}
\date{\nth{25} August 2023 - Present}

\let\stdsection\section
\renewcommand\section{\newpage\stdsection}

\usepackage[parfill]{parskip}

\begin{document}
\maketitle

\tableofcontents

\section{About}
This document is Attila Matolcsy's own personal notes from the University of Gothenburg's Software Engineering and Management Bsc. Programme's Mathematical Foundations for SEM course.

\subsection*{Introduction day}
Lessons are not mandatory, nor are the TA lessons. We can choose our TA groups we'd like to work in.

Working on projects and assignments are also allowed in groups of up to 3 members.

On Canvas there are educational materails, course literatures (although not mandatory) and more information is available.

The teacher can be contacted through e-mail at \href{mailto:christian.berger@gu.se}{christian.berger@gu.se}.


\section{Logic}

Logic means the meaning of mathematical statements and basis of reasoning.\\
We have applications, in the design of computing systems we usually use $1$s and $0$s.

Proofs are mathematical arguments and these are essenctial for programs.\\
We use proofs in computer security systems as well.

Theorems are proven mathematical statements and propositions are statements that are either True (T) or False (F).\\
For propositions we commonly use lettering starting from $p$, so $p,q,r,s,\dots$


\begin{enumerate}[label=Def. \arabic*:, leftmargin=3.5em, align=left]
  \item Opposites: The opposite of the proposition ($T \rightarrow F \And F \rightarrow T$)\\Notation: $\neg$
  \item Conjuction: logical AND, the output of a conjuction is only true when the input statements are all true.\\
  Notation: $\wedge$
  \vspace{.25cm}\\
  \begin{tabular}{cc|c}
    $p$ & $q$ & $p \wedge q$ \\ \hline 
    $T$ & $T$ & $T$ \\
    $T$ & $F$ & $F$ \\
    $F$ & $T$ & $F$ \\
    $F$ & $F$ & $F$ \\
  \end{tabular} \qquad
  \begin{tabular}{cc|c}
    $p$ & $q$ & $p \wedge q$ \\ \hline 
    $1$ & $1$ & $1$ \\
    $1$ & $0$ & $0$ \\
    $0$ & $1$ & $0$ \\
    $0$ & $0$ & $0$ \\
  \end{tabular}
  \item OR: logical OR, the output of a logical OR is true when at least one of the input statements are true.\\
  Notation: $\vee$
  \vspace{.25cm}\\
  \begin{tabular}{cc|c}
    $p$ & $q$ & $p \vee q$ \\ \hline 
    $T$ & $T$ & $T$ \\
    $T$ & $F$ & $T$ \\
    $F$ & $T$ & $T$ \\
    $F$ & $F$ & $F$ \\
  \end{tabular} \qquad
  \begin{tabular}{cc|c}
    $p$ & $q$ & $p \vee q$ \\ \hline 
    $1$ & $1$ & $1$ \\
    $1$ & $0$ & $1$ \\
    $0$ & $1$ & $1$ \\
    $0$ & $0$ & $0$ \\
  \end{tabular}
  \item Excl.OR: exclusive OR, the output of an exclusive OR is true when only one of the input statements are true.\\
  Notation: $\oplus$
  \vspace{.25cm}\\
  \begin{tabular}{cc|c}
    $p$ & $q$ & $p \oplus q$ \\ \hline 
    $T$ & $T$ & $F$ \\
    $T$ & $F$ & $T$ \\
    $F$ & $T$ & $T$ \\
    $F$ & $F$ & $F$ \\
  \end{tabular} \qquad
  \begin{tabular}{cc|c}
    $p$ & $q$ & $p \oplus q$ \\ \hline 
    $1$ & $1$ & $0$ \\
    $1$ & $0$ & $1$ \\
    $0$ & $1$ & $1$ \\
    $0$ & $0$ & $0$ \\
  \end{tabular}
  \item Implication: implication (AKA: If this then this), the output of an implication is false when the first statement is true, but the implied statement is false. In other cases it's true.\\
  Notation: $\Rightarrow$
  \vspace{.25cm}\\
  \begin{tabular}{cc|c}
    $p$ & $q$ & $p \Rightarrow q$ \\ \hline 
    $T$ & $T$ & $T$ \\
    $T$ & $F$ & $F$ \\
    $F$ & $T$ & $T$ \\
    $F$ & $F$ & $T$ \\
  \end{tabular} \qquad
  \begin{tabular}{cc|c}
    $p$ & $q$ & $p \Rightarrow q$ \\ \hline 
    $1$ & $1$ & $1$ \\
    $1$ & $0$ & $0$ \\
    $0$ & $1$ & $1$ \\
    $0$ & $0$ & $1$ \\
  \end{tabular}
  \item Biconditional statements: bidirectional implication, the output of a biconditional statement is true when the both of the propositions have the exact same value\\
  Notation: $\Leftrightarrow$
  \vspace{.25cm}\\
  \begin{tabular}{cc|c}
    $p$ & $q$ & $p \Leftrightarrow q$ \\ \hline 
    $T$ & $T$ & $T$ \\
    $T$ & $F$ & $F$ \\
    $F$ & $T$ & $F$ \\
    $F$ & $F$ & $T$ \\
  \end{tabular} \qquad
  \begin{tabular}{cc|c}
    $p$ & $q$ & $p \Leftrightarrow q$ \\ \hline 
    $1$ & $1$ & $1$ \\
    $1$ & $0$ & $0$ \\
    $0$ & $1$ & $0$ \\
    $0$ & $0$ & $1$ \\
  \end{tabular}
  \newpage
  \item Tautology, A tautology is when the statement is always true, regardless of the propositions' values.\\
  e.g.: $p \vee \neg p \equiv T \equiv 1$
  \item Contardiction, A contradiction is when the statement is always false, regardless of the propositions' values.\\
  e.g.: $p \wedge \neg p \equiv F \equiv 0$
  \item Contingency:
  It's neither tautology nor contradiction.\\The output proposition is not related to the input propositions.
  \item Equivalents:\\
  Statements that are equal\\
  $p \wedge T \equiv p \quad p \wedge 1 \equiv p$\\
  $p \vee T \equiv T \quad p \vee 1 \equiv 1$\\
  $p \wedge F \equiv F \quad p \wedge 0 \equiv 0$\\
  $p \vee F \equiv p \quad p \vee 0 \equiv p$
    
\end{enumerate}



\end{document}